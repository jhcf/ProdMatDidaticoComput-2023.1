\chapter{Tarefa 2: Criar um item no glossário deste documento \label{tarefa:glossario}}

\section{Instruções}

\subsection{Fundamentos}

Um glossário é uma lista alfabeticamente ordenada de termos de difícil compreensão, ou significado técnico especializado, pertinentes. Atua como instrumento de auxílio à compreensão de 
um determinado conjunto de conhecimentos ou documentos.
 
\subsection{A Tarefa}

Registre, no glossário do relatório da disciplina, uma definição de um ou mais termos relacionados com o conteúdo da discussão sobre o que é o estágio.

O texto do item de glossário que você vai criar deve ser escrito em língua portuguesa, deve estar claro e compreensível, e conter uma definição para o termo e, quando pertinente, um exemplo de caso concreto.

A definição deve referenciar pelo menos um item bibliográfico presente no Grupo Zotero ESLC, que contenha título, autor e data.


O diretório onde o arquivo com a tarefa deve ser criado é: 
\begin{verbatim}
1-Introducao /tarefas /T2-Glossario /estudantes    
\end{verbatim}

O nome do arquivo á ser criado deve ter a forma:
\begin{verbatim}
tarefa-<githubusername>.tex
\end{verbatim}

O texto do arquivo deve seguir o modelo da tarefa exemplo do professor, em:
\begin{verbatim}
1-Introducao /tarefas /T2-Glossario /estudantes /tarefa-jhcf.tex
\end{verbatim}

Depois de criar o texto do arquivo com o seu item de glossário,faça o input do mesmo no arquivo:
\begin{verbatim}
1-Introducao /tarefas /T2-Glossario /estudantes /main.tex
\end{verbatim}

\subsection{Compilando a Tarefa}

Após fazer o \textit{input} do conteúdo produzido por você, faça os ajustes necessários para garantir que a compilação do \LaTeX~ ocorre sem introdução de erros ou novos \textit{warnings}.

\subsection{Pontuação pela execução da  Tarefa}

A sua resposta a essa atividade vale até 3\% da pontuação total da disciplina.
